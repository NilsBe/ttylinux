% ttylinux pc_i486-10.0 User Guide
% Copyright (C) 2008-2012 Douglas Jerome <douglas@ttylinux.org>
%
% This document is derived from the ttylinux 7.0 User Guide written
% by Pascal Schmidt; most text copyright (C) Pascal Schmidt.

% $RCSfile:$
% $Revision:$
% $Date:$

% *****************************************************************************
% Preamble
% *****************************************************************************

\documentclass[10pt]{article}

% Use some packages
%
% Fedora Core 10: find packages in /usr/share/texmf/tex/latex/... Try a
% command like:
% $ find /usr/share/texmf/tex/latex/ -name "url.*"
%
\usepackage[headings]{fullpage}
\usepackage{fancyhdr}
\usepackage{listings}
\usepackage{url}
\ifnum\pdfoutput>0
	\typeout{=> Using pdftex.}
	\usepackage[pdftex]{color}
\else
	\typeout{=> Using dvips; pdfoutput is not defined.}
	\usepackage[dvips]{color}
\fi

% Setup the page characteristics.
%
\pagestyle{fancy}
\setlength\topmargin{-0.50in}
\setlength\textheight{8.75in}
\parindent=0pt
\parskip=0pt

% Setup header and footer using the fancyhdr package.
%
\renewcommand\headrulewidth{0.4pt}
\fancyhead{}
\fancyhead[R]{\thepage}
\fancyhead[L]{\nouppercase{\leftmark}}
\renewcommand\footrulewidth{0.4pt}
\fancyfoot{}
\fancyfoot[R]{\thepage}
\fancyfoot[L]{ttylinux : pc\_i486 10.0}
\footskip=48pt

% Sans serif typeface is the default.
%
\renewcommand\familydefault{\sfdefault}

% These colors are intended for listing package use.
%
\definecolor{lightblue}{rgb}{0.80,0.80,1.00}
\definecolor{lightgray}{rgb}{0.90,0.90,0.90}
\definecolor{darkgray}{rgb}{0.40,0.40,0.40}

% Setup listings package.
%
\lstset{language=bash}
\lstset{rulesepcolor=\color{darkgray}}
\lstset{aboveskip=10pt}
\lstset{belowskip=0pt}
\lstset{basicstyle=\ttfamily\small}

\title{ttylinux User Guide}
\author{pc\_i486 10.0\\ \\Maintained by Douglas Jerome\\Based on Previous Work by Pascal Schmidt}

% *****************************************************************************
% Title Page
% *****************************************************************************

\begin{document}

\maketitle
\thispagestyle{empty}
\newpage

% *****************************************************************************
% Table of Contents
% *****************************************************************************

\pagenumbering{roman}
\tableofcontents

% *****************************************************************************
% Document Body
% *****************************************************************************

\newpage
\pagenumbering{arabic}
\parskip=10pt
\section{Introduction}

ttylinux is a small, minimal Linux distribution. It is freely available as a
bootable CD-ROM image. The build system of shell scripts and configuration
files that build the bootable CD-ROM image is also freely available.

This document provides information about using and installing ttylinux. The
audience of this document should be comfortable with using the Bash command
line.

The word ttylinux has no capital letters, ever. "TTYlinux", "TTY-Linux",
"TtyLinux", "Ttylinux" and all other usages of a capital letter or extra symbol
are wrong. When spoken, ttylinux sounds like "t - t - y - linux".

Using the ttylinux build distribution for building ttylinux is beyond the scope
of this document. The build distribution has a short text file describing how
to build ttylinux. When available, the Developer Guide describes building
ttylinux.

This User Guide is primarily for the {\it PC} platform ttylinux availble for
the {\bf i486}, and is somewhat applicable to the {\bf i686} and {\bf x86\_64}
PC variants. This User Guide is also usefull for understanding the basics of
the other ttylinux variants.

\subsection{ttylinux Overview}

ttylinux tries to use as little space as possible and be a familiar and
complete command-line Linux system, with fairly up-to-date Linux kernel and
program utilities. It provides multi-tasking, multi-user Linux with networking
capabilities in no more than 8 MB of disk space. It is prepared for Internet
access by Ethernet. A text-based web browser, command-line remote login secure
client and server, NFS client, FTP server, and TFTP server are a part of
ttylinux.

ttylinux can be installed onto a disk drive, both spinning hard drive or flash
drive such as a USB memory stick; it can be manually installed or installed by
using an installer script. Installation by installer script or manual
installation can be done with ttylinux itself or by using a different Linux
system.

The ttylinux file system, excluding the Linux kernel, is 8 MB in size. Using
the ttylinux installer script, a Linux kernel between 2.5 and 3.5 MB will also
be installed. This makes a minimum workable size of about 12 MB for a hard
drive partition on which to install ttylinux, although 32 MB of RAM are needed
to use the automated ttylinux installer script.

ttylinux includes a package management script, named {\bf pacman}, capable of
{\it installing}, {\it making}, {\it removing} and {\it querying} ttylinux
packages and their files. The package manager can {\it list} and {\it install}
packages from an external repository via http. Pacman is useful for adapting
ttylinux to specific needs.

{\bf What ttylinux Is Not}

ttylinux is not a typical Linux distribution; it does not have a graphical user
interface, software development tools, music player, document preparation nor
printing tools, databases nor network services such as BIND, News Server nor
Mail Transfer Agents.

{\bf What ttylinux Can Do}

ttylinux is intended to be useful as the basis of an embedded system or a
directed-purpose system: with its small size ttylinux boots quickly from flash
drives and CR-ROM; it has been used as a system fix/repair tool, as a
simulation host, and is a good basis for a rescue or installation CD-ROM.

ttylinux provides a working Linux environment with its boot image, and custom
task-specific scripts can mount other parts of the file system to provide a
larger system.

ttylinux is useful on computers which are considered obsolete, such as 486SX
PC. It is for people who want to have a minimal Linux distribution to run when
little space is available or needed. Some users may want to use the ttylinux
file system but configure and build their own Linux kernel.

ttylinux can serve as a rough prototype of a larger system, since it uses the
same C library, glibc, as full Linux systems, compiling programs on a different
Linux computer and copying them to the ttylinux file system \underline{can}
result in working programs. {\bf However this is not a supported feature.}
Programs compiled outside the normal ttylinux build process may require
libraries not present in ttylinux. Worse, they may be compiled on a computer
with different Linux kernel capabilities and make system calls not present in
ttylinux.

ttylinux is for people who have Linux experience; it is not for beginners,
unless you want to learn how a Linux system works underneath the Graphical User
Interfaces. You must be able to use the interactive shell command-line, and it
helps to know your way around Linux system. Most of the programs are
{\bf busybox} programs; these are smaller versions of the common Unix utilities.

\subsection{Licenses}

The software packages that are part of ttylinux are licensed under a number of
different open source licenses, as listed below. The initialization and system
service scripts developed by the ttylinux project are licensed under the GNU
General Public License; a copy of this license is included in the file
\url{COPYING}.

\begin{center}
\begin{tabular}{l|l|l}
Package      & Version & License \\
\hline
bash         & 4.2    & GPL  \\
busybox      & 1.19.3 & GPL  \\
dropbear     & 0.53.1 & MIT  \\
e2fsprogs    & 1.42   & GPL  \\
glibc        & 2.9    & LGPL \\
iptables     & 1.4.12 & GPL  \\
lilo         & 23.2   & BSD  \\
ncurses      & 5.7    & MIT  \\
ppp          & 2.4.5  & GPL  \\
retawq       & 0.2.6c & GPL  \\
\end{tabular}
\end{center}

For more information on the licenses, please visit the \url{opensource.org}
website.

\newpage
\section{Starting with ttylinux}

This section has a general overview of the ttylinux download CD-ROM image and
also describes the system hardware requirements for using ttylinux, from where
to download ttylinux, what to download and how to use the downloaded images.

ttylinux has three basic parts: a boot loader, a Linux kernel, and a root
file system. All three of these are in the CD-ROM image; the CD-ROM image can
be burned onto a blank CD-ROM disc and then booted. When booted, the root
file system from the CD-ROM is decompressed and becomes a read/write root file
system in a RAM disk in memory. Note that changes to any of the files while
running ttylinux are lost, as they are in a RAM disk. Booting the ttylinux
CD-ROM is further described in section \ref{bootcd}.

Installing ttylinux from the bootable CD-ROM onto a hard drive is described in
section \ref{bootfixed}. Installation onto a hard drive makes a system
different from the bootable CD-ROM; the installed ttylinux has a read/write
root file system directly on a spinning hard disk or solid state disk, not in a
RAM disk. The advantage of a hard drive ttylinux system over the RAM disk
system is that file changes are not lost.

ttylinux can be put onto a flash drive, such as a USB drive, which can be made
bootable. This copies the RAM disk boot method to the flash drive; when the
flash drive is booted, the root file system from the flash drive is decompressed
and becomes a read/write root file system in a RAM disk in memory. As with the
system booted from CD-ROM, the changes to files are lost when the system shuts
down. The process of putting ttylinux onto a flash drive is described in more
detail in section \ref{bootflash}.

The ttylinux root file system is a compressed image file on the CD-ROM; it can
be copied and used with a different custom kernel, one that you make, and put
onto other media with your boot loader of choice. This process is beyond the
scope of this document, but the requirements for a ttylinux custom kernel are
described in more detail in section \ref{customkernel}.

\subsection{System Requirements}

ttylinux is available for several different CPUs and hardware architectures;
currently, ttylinux operation on a PC compatible i486 architecture is described
in this User Guide, with much applicability to the i686 and x86\_64 PC
platforms. These systems are:\\
{\tt ttylinux 10.0} - {\bf i486}, specifically the i486 instruction set\\
{\tt ttylinux 10.0} - {\bf i686}, Pentium Pro instruction set\\
{\tt ttylinux 10.0} - {\bf x86\_64}, AMD64/Intel 64 instruction set

{\bf CPUs and Computers}

pc\_i486 ttylinux requires an i486SX or newer processor in a PC compatible
computer. It will not work with the i386 CPU; the glibc version in ttylinux
uses CPU instructions the i386 CPU does not have. Any x86 compatible CPU
supporting i486, and upward compatible, that is in a PC compatible computer
should work.

pc\_i686 ttylinux requires Pentium Pro or newer processor in a PC compatible
computer. Any x86 compatible CPU supporting Pentium Pro, and upward compatible,
that is in a PC compatible computer should work.

pc\_x86\_64 ttylinux supports generic 64-bit AMD64/Intel 64 processor in a PC
compatible computer.

{\bf Memory}

ttylinux uses an 8 MB RAM disk when booted from CD-ROM. The kernel on the
CD-ROM is fairly large; it supports a broad range of hardware, so at least 28
MB of memory are required for full operation with pc\_i486 ttylinux and at
least 128 MB of memory are required for full operation with pc\_i686 or
pc\_x86\_64 ttylinux.

Using a custom kernel supporting only hardware for a particular computer, an
i486 ttylinux system may require as little as 16 MB of memory to run. If the
file system is installed onto a read/write disk drive, spinning or flash, and a
custom kernel is used, an i486 ttylinux will run within 8 MB of RAM.

\subsubsection{Custom Kernel Requirements}
\label{customkernel}

The ttylinux root file system is an 8 MB ext2 file system; the file system
image is compressed and resides in the CD-ROM image. After burning the CD-ROM
image to a blank CD-ROM disc, or mounting the CD-ROM image via loop device,
you can find the compressed root file system; it is \url{boot/filesys.gz}.
This root file system can be used with a custom kernel that you make.

pc\_i486, pc\_i686, and pc\_x86\_64 ttylinux are built with
{\bf Linux 2.6.34.6} header files.  Linux kernels are not backwards compatible;
software using the capabilities of a given kernel version cannot be
{\it expected} to work with any previous kernel version. Using a kernel older
than the kernel with which ttylinux was built cannot be supported in any way.
With that described, with the small number of packages in the ttylinux system,
ttylinux works to some extent with any Linux kernel from 2.6.0 upwards.

Your custom kernel needs to support all the hardware you want to use, plus some
additional requirements for ttylinux itself.

A kernel used for running ttylinux needs to have {\bf ramdisk} support, {\bf
initial ramdisk} support, and a default ramdisk size of at least {\bf 8192}.
Note the kernel configuration has a default ramdisk size of 4096, which is not
big enough.

If you want to use the basic firewall script of ttylinux, your kernel also
needs iptables support with the {\bf netlink} interface.

A ttylinux kernel needs to support {\bf ext2 file systems}.

If you want to add a telnet server to ttylinux, your kernel will need to have
{\bf Unix98 pseudo terminal} support and support for the {\bf devpts} file
system.

\subsection{File Downloads}

The main ttylinux web site is accessed at \url{http://ttylinux.org/}, and
\url{http://www.ttylinux.net/} should have the same content.

The ttylinux web site has a Download page that has several files available for
downloading.

{\bf Bootable CD-ROM Images}

There should be several ttylinux bootable CD-ROM ISO images available, at least
one each for i486 PC, i686 PC, and x86\_64 PC. The CD-ROM ISO images are each
an El Torito bootable CD-ROM ISO 9660 file system with the Joliet and Rock
Ridge extensions. El Torito enables CD-ROM to be bootable on PC. The Joliet and
Rock Ridge extensions add longer file names to the ISO 9660 file system
capabilities.

{\bf Source Code}

The ttylinux web site should have links to the sources of the source code
packages used to build ttylinux. ttylinux source ISO distributions, which have
the source code packages comprising the ttylinux variants, are probably
available at the ttylinux web site. If you cannot find the source code package
for a ttylinux component, then email \url{douglas@ttylinux.org}, and any
needed arrangements will be made to supply the source code package.

{\bf Build System Distribution}

The complete ttylinux build system distribution is available; it has a file
\url{How_To_Build_ttylinux.txt} that describes the build process. When
available at the ttylinux web site, the Developer Guide more fully describes
building ttylinux.

{\bf Binary Run-time Packages}

The binary packages that make up the entire ttylinux run-time system are
available within the distribution ISO, or distribution TAR-file. These packages
are available in the case any were removed from a ttylinux system and there is
a desire or need to reinstall the removed packages. Packages are installed with
\url{pacman}, the ttylinux package manager. Pacman is described in more detail
in section \ref{pacmanger} of this document.

\subsection{Booting a CD-ROM Image}
\label{bootcd}

The PC variants of ttylinux are intended to boot on an appropriate 32-bit x86
or 64-bit x86\_64 PC that can boot from a CD-ROM drive.

Download the CD-ROM ISO image file and burn it onto a blank CD-ROM disc as an
ISO image. Then put the disc into the CD-ROM drive of an appropriate PC and
boot the PC; ttylinux should start up automatically.

A computer's BIOS setup may not be set up to allow booting from CD-ROM; in that
case you need to go into the BIOS setup screen(s) and make changes that allow
the computer to boot from CD-ROM. If the computer has an old BIOS that is not
able to boot from a CD-ROM device, there is software called {\it Smart Boot
Manager} that may help. It can currently be found at:
\url{http://btmgr.sourceforge.net/about.html}

Once ttylinux has booted, and you see the login prompt, login as user name
"root", the administrator account, with password "password". There also is a
non-administrator account "user", with password "password".

The CD-ROM can be used as a rescue system or simply for trying ttylinux. For
installing or transferring ttylinux from the CD-ROM, or from the downloaded
CD-ROM image file, to another disk device see section \ref{installation}. See
section \ref{sysguide} of this user guide for pointers about what you can do
with a ttylinux system.

\subsection{Setting Up a USB or Flash Drive}

ttylinux can be put onto a USB drive, also known as flash drive, USB memory
stick, pen drive, travel drive, etc. This also applies to flash drives that are
not on USB. For these you probably want to boot a RAM disk system from your USB
or flash drive. See section \ref{bootflash}, but you should also read the
preceeding parts of section \ref{installation}.

\newpage
\section{Installing from CD-ROM}
\label{installation}

This section of the user guide describes the two types of ttylinux bootable
installations and the methods for creating them from either the downloaded
CD-ROM image file or a CD-ROM with the image burned onto it.

\subsection{CD-ROM Image Overview}

The ttylinux CD-ROM image is the source used for installation; it is a CD-ROM
ISO 9660 file system. The pc\_i486 CD-ROM has the following directory structure:

\begin{lstlisting}
|-- AUTHORS
|-- COPYING
|-- LABEL
|-- boot
|   |-- System.map
|   |-- filesys.gz
|   |-- grub
|   |   `-- loopback.cfg
|   |-- isolinux
|   |   |-- boot.msg
|   |   |-- help_f2.msg
|   |   |-- help_f3.msg
|   |   |-- help_f4.msg
|   |   |-- isolinux.bin
|   |   `-- isolinux.cfg
|   |-- vmlinux
|   `-- vmlinuz
|-- config
|   |-- kernel-2.6.34.6.cfg
|   |-- syslinux
|   `-- ttylinux-setup
|-- doc
|   |-- ChangeLog-pc_i486
|   |-- Flash_Disk_Howto.txt
|   |-- User_Guide.html
|   |-- User_Guide.pdf
|   `-- User_Guide.tex
|-- packages
|   |-- bash-4.2-i486.tbz
|   |-- busybox-1.19.3-i486.tbz
|   |-- dropbear-0.53.1-i486.tbz
|   |-- e2fsprogs-1.42-i486.tbz
|   |-- glibc-2.9-i486.tbz
|   |-- gpm-1.20.6-i486.tbz
|   |-- iptables-1.4.12-i486.tbz
|   |-- lilo-23.2-i486.tbz
|   |-- ncurses-5.7-i486.tbz
|   |-- ppp-2.4.5-i486.tbz
|   |-- retawq-0.2.6c-i486.tbz
|   |-- ttylinux-basefs-1.0-i486.tbz
|   |-- ttylinux-devfs-1.0-i486.tbz
|   `-- ttylinux-utils-1.3-i486.tbz
`-- qemu-i486.sh
\end{lstlisting}

Several files are critical for installation, note their location in the CD-ROM
image:\\
{\tt CD-ROM/}{\bf boot/filesys.gz} -- ttylinux gzipped ext2 file system image\\
{\tt CD-ROM/}{\bf boot/vmlinuz} -- gzipped ttylinux Linux kernel\\
{\tt CD-ROM/}{\bf config/ttylinux-setup} -- user-maintained RAM disk startup

The ttylinux installation script automates the process of installing ttylinux;
the scripts also copy the other documentation and information files to the new
installed system.

For manual installation you can mount the CD-ROM or mount the CD-ROM image file
via loop device for access to the critical files. The manual processes of
installing ttylinux describes this in more detail.

\subsection{RAM Disk or Persistent Storage Boot}

There are two basic types of ttylinux installation, resulting in two type of
booted systems: a RAM disk or persistent storage.

A {\bf Ram Disk} installation results in a system that puts the root file
system into RAM when it boots, which is what the bootable CD-ROM does. If you
want to put ttylinux onto a flash drive, pen drive, USB memory stick, travel
drive, etc. then you very probably want this sort of installation. This is
typical for flash drives, but not for spinning hard disks, for at least two
reasons: 1) flash drives have been much slower than hard drives, so maintaining
a live file system on a flash drive has been intolerably slow, and 2) these are
removable drive which have been difficult to consistently mount as they move
between interfaces and computers. With this booting scheme, the boot loader
takes the root file system from the drive and gives it to the kernel which
decompresses it and mounts it as a read/write root file system in a RAM disk in
memory. Changes to files in the root file system are lost when the system shuts
down; persistent changes must be stored elsewhere. However, ttylinux has
specific support for persistent changes to its boot-time startup with a RAM
disk system, this is described section \ref{flashback}. The program
\url{/sbin/ttylinux-flash} can be used to copy ttylinux from the CD-ROM to
another drive and configure it to boot in this manner. The processes of putting
ttylinux onto a drive for booting a RAM disk system and configuring its
boot-time startup support is described in more detail in the next section
\ref{bootflash}.

A {\bf Persistent Storage} installation results in a system that boots with the
read/write root file system maintained directly on the spinning hard disk or
solid state drive. If you have a spinning hard disk or one of the new fast
solid state drives, and it is not removable, then you probably want this sort
of installation. If you want to install onto removable media and will move the
media to different slots or computers, then you do not want this kind of
installation. The program \url{/sbin/ttylinux-installer} can be used to install
ttylinux from the CD-ROM onto a spinning hard disk or solid state drive and
configure it to boot in this manner. The processes of installing ttylinux onto
a hard drive and booting a persistent storage root file system is described in
more detail in section \ref{bootfixed}.

\subsection{Transfer from CD-ROM -- Make a RAM Disk Boot System}
\label{bootflash}

The section describes transferring a few files from the CD-ROM image to a drive,
probably a flash drive, and configuring it to boot ttylinux into a RAM disk, as
does the CD-ROM.

A Linux system, either ttylinux or some other Linux system, can be used to make
a ttylinux bootable flash drive.

The ttylinux script \url{/sbin/ttylinux-flash} makes a bootable ttylinux on a
flash drive, and it does this in such a way that the new flash drive ttylinux
copy can then be used in place of a CD-ROM as the {\it source} for another
ttylinux transfer, but only if you again use the ttylinux script.

\subsubsection{Source Directory}

The ttylinux CD-ROM is used as the source, or the ttylinux CD-ROM image file
mounted with a loop device can be used. Even the kernel and file system image
files removed from the ttylinux CD-ROM image can be used as the source if they
are in a directory structure as found in the CD-ROM image.

{\bf Using the ttylinux CD-ROM Disc}

For the following examples of mounting the CD-ROM disc, /mnt/cdrom references
the mount point in your file system to which the CD-ROM disc mounts. If you are
not using ttylinux then your actual mount point may be different; substitute
accordingly. Have the ttylinux boot CD-ROM disc in the CD-ROM drive and mount
it. You need to know which device in /dev to use; if you do not know which
device to use then section \ref{cdromdev} might help. The CD-ROM should be
mounted as type iso9660 e.g., mounted by the following command.

\begin{lstlisting}
	mount -t iso9660 /dev/<partition> /mnt/cdrom
\end{lstlisting}

{\bf Using a ttylinux CD-ROM ISO Image File}

If you have a downloaded ttylinux CD-ROM image file, ttylinux-i486-10.0.iso.gz,
ttylinux-i686-10.0.iso.gz or ttylinux-x86\_64-10.0.iso.gz, then you can mount
it via loopback device with the following commands; substitue i686 or x86\_64
for i486 where appropriate.

\begin{lstlisting}
	mkdir -p mnt/cdrom
	gunzip ttylinux-i486-10.0.iso.gz
	mount -t iso9660 -o loop ttylinux-i486-10.0.iso mnt/cdrom
\end{lstlisting}

{\bf Critical ttylinux CD-ROM Files}

Of the following three files, you must have access to the first two; the third
one is very usefull, but not critical. In the following, "{\tt CD-ROM/}" is
meant to be wherever you mounted the CD-ROM disc or CD-ROM image file.

{\tt CD-ROM/}{\bf boot/filesys.gz} -- ttylinux gzipped ext2 file system image\\
{\tt CD-ROM/}{\bf boot/vmlinuz} -- gzipped ttylinux Linux kernel\\
{\tt CD-ROM/}{\bf config/ttylinux-setup} -- user-maintained RAM disk startup

\subsubsection{Target Directory}

The target directory is the directory where ttylinux will be put; it must be
the top-level, root directory on the disk partition being used. You need to
know, or find out, the device name for the disk partition onto which you want
to transfer ttylinux. If you are not sure what a disk partition is you can read
a little more description of ttylinux target partitions in section
\ref{targetpartition}, but do not continue until you understand enough about
disk devices and partitions to understand the rest of this section.

Due to the combined space requirements of the 8 MB ttylinux file system and the
2.5 to 3.5 MB ttylinux Linux kernel, and considering some margin, the minimum
partition size onto which you can install ttylinux and have it work is at least
12 MB. These sizes are much larger for the i686 and x86\_64 ttylinux variants.

This rest of this section describes manually mounting a disk partition that has
the directory to transfer ttylinux onto. If your target directory is already
mounted, or automatically mounts, and you will use the ttylinux script to
transfer ttylinux, then {\bf delete everything in the target directory} or the
script will not transfer ttylinux onto it.

In order to manually mount a disk you need to know the disk device node e.g.
/dev/sdc and its mountable partition you want to use e.g. /dev/sdc1. Read the
previous sentence again, note the distinction between the disk and partition
devices.

A USB drive partition probably should be mounted with the following command.
For the following example of mounting the drive, /mnt/flash references the
mount point in your file system to which the drive mounts. If you are not using
ttylinux then your actual mount point may be different; substitute accordingly.
If you are mounting a Linux file system then change to the appropriate file
system type in the following command.

\begin{lstlisting}
	mount -t vfat /dev/<partition> /mnt/flash
\end{lstlisting}

The device partition in the above example is the device node of the mountable
partition on the disk that you want to use e.g. sdc1, in which case it
represents /dev/sdc1.

If you will use the ttylinux script to transfer ttylinux, then {\bf delete
everything in the target directory} or the script will not transfer ttylinux
onto it.

\subsubsection{Running the Transfer Script}

The ttylinux shell script, \url{/sbin/ttylinux-flash} automates the process of
copying the ttylinux system from the source directory into the target directory
and making the target drive bootable. This transfer typically is from CD-ROM
disc to flash drive; the CD-ROM disc should be mounted with option {\tt -t
iso9660} to specify the correct file system type, and USB drives are usually
FAT32 and those should be mounted with option {\tt -t vfat} to specify the
correct file system type.

The script is invoked with a command line option telling it which boot loader
to use, lilo or syslinux. The following is the help output from the
\url{ttylinux-flash} script, it describes how to invoke the script.

\begin{lstlisting}
ttylinux-flash
(C) 2008-2010 Douglas Jerome <douglas@ttylinux.org>

Usage: ttylinux-flash --lilo     <source path> <flash path> <flash dev>
       ttylinux-flash --syslinux <source path> <flash path> <flash dev>

Parameters:

     -l | --lilo ....... Use lilo method to make bootable flash disk.
     -s | --syslinux ... Use syslinux method to make bootable flash disk.

     <source path> is the mounted ttylinux CD-ROM directory, or any equivalent
                   USB or hard drive directory of the ttylinux CD-ROM layout
                   and contents.

     <flash path>  is a rooted path to the flash disk root file system to be
                   loaded from the source path.  For the syslinux method this
                   must be a Windows FAT file system, but for the lilo method
                   this can be either an EXT2 or FAT file system.

     <flash dev>   is the /dev/* that is the whole disk block device node,
                   such as /dev/sdc, NOT a partition block device node
                   like /dev/sdc1.
\end{lstlisting}

The transfer script checks if the source CR-ROM device contains a ttylinux
CD-ROM; if the CD-ROM is found then a summary of what is to be transfered onto
which device is printed and you are given a choice of continuing or aborting.
Enter "yes" to continue the transfer.

The transfer script copies the ttylinux files onto drive and then installs the
bootloader.

After the installer is finished you can remove the CD-ROM disc from your
computer and reboot.

\subsubsection{Manual Transfer}

The manual process is described in appendix \ref{flashdiskhowto} of this
document. It also is a text file, \url{Flash_Disk_Howto.txt}, in the doc/
directory on the ttylinux CD-ROM disc.

\subsubsection{Configuring the System}
\label{flashback}

This transfered ttylinux system is a RAM disk system. Its file system is a
gzipped image on the boot drive; its directories and files are very
inaccessible. When the system is running, changes made to files are not
retained for the next boot. Customizing or configuring this system to your
needs takes some specific support.

A ttylinux system properly transfered, such as by using the
\url{/sbin/ttylinux-flash} script, has a special boot-time startup feature:
when ttylinux boots it attempts to mount the USB/flash drive that is boots from
and run a script on that drive. Since the script is in a boot drive directory,
and not burried away in a compressed file system, it can be changed and
maintained by you. You can change this script to perform ttylinux configuration
for ttylinux to do when it boots.

The feature of mounting the booted drive and running a startup script on that
drive provides the user with a persistent boot-time startup configuration that
is retained from one boot to the next. This user-maintainable startup script on
the boot drive is in the \url{config/ttylinux} directory; a default version of
this script is put onto the boot drive when the \url{/sbin/ttylinux-flash}
script is run. This startup script is a very convenient place to configure your
particular network interface upon ttylinux startup.

\subsection{Install from CD-ROM -- Make a Persistent Storage Boot System}
\label{bootfixed}

{\bf NOTE} Your computer needs at least 32 MB of RAM to run the
\url{ttylinux-installer} script.

The section describes installing ttylinux from the CD-ROM image to a drive
partition, probably on a spinning hard disk, and configuring it to boot
ttylinux with the root file system residing on the boot drive, not in RAM.

A Linux system, either ttylinux or some other Linux system, can be used to
install ttylinux.

WARNING: Running the installer can easily destroy all operating systems, and
anything else, currently present on the target machine. Proceed with caution
and backup all {\it important} data before installing ttylinux. Really.

To install ttylinux onto disk from the bootable CD-ROM, you first need to burn
the ttylinux CD-ROM ISO image onto a blank CD-ROM disc and, if using ttylinux
to perform the installation, boot into it as described in the previous section
\ref{bootcd}.

Once logged into ttylinux as root, you can start the installation. You need to
know three things to run the installer: 1) what your CD-ROM device is, 2) which
drive partition you want to install ttylinux, and 3) where you want to put the
boot loader.

If you don't know the answers to those three questions after reading the
following instructions, the safe bet would be {\bf not} to proceed with
installation; the ttylinux installer is not yet automated or user-friendly
enough for you.

\subsubsection{Source CD-ROM Device}
\label{cdromdev}

The correct CD-ROM device name depends on whether the drive is an IDE or SATA
device. If your system uses IDE, the following device names are possible:

\begin{center}
\begin{tabular}{l|l}
Device Name & Description \\
\hline
/dev/hda & Master Device on First IDE Controller  \\
/dev/hdb & Slave Device on First IDE Controller   \\
/dev/hdc & Master Device on Second IDE Controller \\
/dev/hdd & Slave Device on Second IDE Controller  \\
\end{tabular}
\end{center}

Among the above, /dev/hda is not likely to be your CD-ROM device unless you are
using a modern laptop. A more likely possibility is /dev/hdc. /dev/hda normally
is the device name of your hard disk, but a modern computer will use SATA for
the hard drive and many of those have IDE CD-ROM drive.

If your system uses SATA (Serial ATA), use this table:

\begin{center}
\begin{tabular}{l|l}
Device Name & Description \\
\hline
/dev/scd0 & First SATA CD-ROM Device  \\
/dev/scd1 & Second SATA CD-ROM Device \\
/dev/scd2 & Third SATA CD-ROM Device  \\
/dev/scd3 & Fourth SATA CD-ROM Device \\
\end{tabular}
\end{center}

Usually the SATA CD-ROM device will be /dev/scd0.

\subsubsection{Target Partition Device}
\label{targetpartition}

You need to know, or find out, the device name for the disk partition on which
you want to install ttylinux. The device names for disk partitions are formed
by appending a number to the device name of the corresponding disk. For
example, if your disk device is /dev/hda, the device /dev/hda3 is the third
partition on that disk. Numbers 1-4 are the primary partitions, extended
partitions start at 5.

ATTENTION: If you plan on installing onto a USB drive, or some other frequently
moved disk device, then do not install ttylinux with the instructions here; use
the instructions in section \ref{bootflash}. The disk and partition devices
used by this installation process would likely be different between different
computers, so this installation may not correctly boot when booted on a
computer other than the computer on which the installation is performed.

Due to the combined space requirements of the 8 MB ttylinux file system and the
3 MB ttylinux kernel, and considering some margin, the minimum partition size
onto which you can install ttylinux and have it work is about 12 MB.

IDE disks use the same device names as given for IDE CD-ROM devices above. For
SATA, the names are as follows:

\begin{center}
\begin{tabular}{l|l}
Device Name & Description \\
\hline
/dev/sda & First SATA Disk Device  \\
/dev/sdb & Second SATA Disk Device \\
/dev/sdc & Third SATA Disk Device  \\
/dev/sdd & Fourth SATA Disk Device \\
\end{tabular}
\end{center}

Note that if you want to create a dual-boot setup with Windows and ttylinux on
the same disk, a topic not covered here, you can't use the first partition
/dev/hda1 or /dev/sda1 as your ttylinux target partition, because that is where
Windows needs to be installed to work.

Here are some examples of possible device names for your target partition:

\begin{center}
\begin{tabular}{l|l}
Device Name & Description \\
\hline
/dev/hda1 & First Primary Partition on Primary IDE Master \\
/dev/hdb5 & First Extended Partition on Primary IDE Slave \\
/dev/sda2 & Second Primary Partition on First SATA Disk   \\
/dev/sdc6 & Second Extended Partition on Third SATA Disk  \\
\end{tabular}
\end{center}

Note that depending on the BIOS, booting might be possible only from the first
two disks installed in the system.

Also, you can look at the directory listing of \url{/sys/block} to see which
block devices the kernel has detected, as disk drives are block devices.

{\bf Drive Partitioning, if Needed}

What to do if your target disk is not partitioned yet? Linux systems, including
ttylinux, have the \url{fdisk} program that can be used to partition disks. For
example, to partition a disk connected as master to the first IDE controller,
use:

\begin{lstlisting}
	fdisk /dev/hda
\end{lstlisting}

The user interface of \url{fdisk} is somewhat primitive, {\it so be careful}.
If you haven't used it before, a good idea would be to search the Internet for
instructions. The basic commands you may need are "d" to delete a partition,
"n" to create a new partition, "p" to print the current partition table, and
"w" to write the edited partition table to disk. You can also use "q" to exit
\url{fdisk} without saving your changes.

\subsubsection{Boot Loader Location}

The LILO boot loader is installed in one of two places: either the Master Boot
Record (MBR) of the {\it disk device} or the boot sector of the {\it partition
device} in which ttylinux is being installed.

With LILO installed in the MBR of the first disk, it will completely take over
the entire boot process of the computer. If there are other operating systems
installed on the computer they need to be specified in the LILO configuration
file, \url{/etc/lilo.conf}, in order to boot them.

With LILO installed in the boot sector of the target partition or in the MBR of
a disk other than the first one in your computer, the bootloader installed in
the MBR of the first disk needs to be configured to boot the ttylinux target
partition.

\subsubsection{Running the Installer}

Once you have decided on target device and boot loader location, you can run
the installer. The script is called \url{ttylinux-installer}. The following is
the help output from the \url{ttylinux-installer} script, it describes how to
invoke the script. The square brackets indicate an optional parameter, the
{\it partition} device is used for the installation target device.

\begin{lstlisting}
ttylinux-installer
(C) 2008-2011 Douglas Jerome <douglas@ttylinux.org>

Usage: $(basename $0) [ <options> ] <source device> <target device>
Usage: $(basename $0) --config=<file> <source device>

Options:
     --help ......... Show this help.
     -m | --mbr ..... Put lilo boot loader onto the Master Boot Record of the
                      disk device containing the <target device> disk partition.
     --nolilo ....... Do not put the lilo boot loader onto the disk drive.
     --vcs=<name> ... Use <name> for the Virtual Context Script.

<source device> is the CD-ROM device that has the ttylinux CD-ROM.

<target device> is the disk partition device onto which ttylinux is installed;
                it needs to be a disk partition device, not the whole disk
                device.  An ext2 file system is created on this device.

Create an ext2 file system on the disk partition device <target device>,
install ttylinux from the CD-ROM <source device> into the new file system on
<target device>, and then maybe put a lilo boot loader onto the target disk.
The lilo boot loader is put onto the disk partition <target device> unless the
-m or --mbr option is present, in which case the it is put onto the Master Boot
Record of the disk device.  No lilo boot loader is put onto the disk if the
--nolilo option is preset or if the running architecture does not support lilo
e.g., ttylinux PowerPC.
\end{lstlisting}

For example, to install from the CD-ROM device /dev/hdc into partition device
/dev/hda2 and placing LILO on the MBR, /dev/hda disk device, you would use:

\begin{lstlisting}
	ttylinux-installer -m /dev/hdc /dev/hda2
\end{lstlisting}

Another example, installing from the second SATA CD-ROM device /dev/scd1 into
the third partition device of the second SATA disk and placing LILO on the boot
sector {\it of the target partition}:

\begin{lstlisting}
	ttylinux-installer /dev/scd1 /dev/sdb3
\end{lstlisting}

The installer checks if the source CR-ROM device contains a ttylinux CD-ROM
disc; if the CD-ROM disc is found then a summary of what is to be installed on
which device is printed and you are given a choice of continuing or aborting.
Enter "yes" to continue the installation.

The installer creates an ext2 file system on the target partition then copies
the ttylinux distribution files onto the new file system, and then installs the
LILO bootloader.

After the installer is finished you can remove the CD-ROM disc from your
computer and reboot.

{\bf Using the Installation Configuration File}

When you boot ttylinux from the CD-ROM image and log in as root, you will find
a file named "install.conf" in the /root directory; this file is an example of
an installation configuration file that can be used to direct the automated
installtion activites of the ttylinux installer script. The example
"install.conf" file is well documented; read it to learn how to use it.

You must first manually partition the hard drive, and then make sure the
installation configuration file's fstab section matches the actual disk
partitions.

Example, to install from the CD-ROM device /dev/hdc into partition device
/dev/hda2, using the installation configuration file:

\begin{lstlisting}
	ttylinux-installer --config=install.conf /dev/hda2
\end{lstlisting}

{\bf Using a Virtual Context Script}

The ttylinux installer script supports a Virtual Context Script option in which
the installed system is locked-down and boots with a startup sequence that
mounts an iso9660 (CD-ROM) file system and runs a contextualization script from
that file system. This is for running the installed ttylinux system in a
virtual environment that may be prone to unauthorized log-in attempts.

The \url{--vcs=<name>} option names the Virtual Context Script to be invoked at
boot startup each time the installed system boots. The named Virtual Context
Script is inkvoked from an iso9660 file system that is mounted at boot startup;
the block device for this file system must be available as /dev/hda, /dev/hdc,
/dev/sr0, or /dev/cdrom.

When the \url{--vcs=<name>} option is used, the installed system is locked down
by:\\
1. the user account is removed\\
2. the getty logins are all disabled\\
3. dropbear (sshd) is configured to run with passwords disabled\\
4. the firewall is enabled allowing only the SSH port\\
5. the Virtual Context Script is configured to be invoked at startup\\
6. the startup is configured to create a random root password

It is expected that the Virtual Context Script, which is invoked each time the
installed system boots, puts an ssh-compliant public key into the
/root/.ssh/authorized\_keys file. This allows root login via SSH by someone
with the corresponding SSH-compliant private key. See section \ref{dropbear}
for some more information on dropbear and public/private key usage.
 
\subsubsection{Manual Installation -- Setup and Installation}

This description uses \url{LILO} for boot loading; other boot loaders such as
\url{grub} and maybe \url{loadlin} and \url{syslinux} will also work.

There are two files to take from the ttylinux CD-ROM image, either by burning
the image to a blank CD-ROM disc and mounting it, or mounting the CR-ROM image
via loop device. In the following commands, the ttylinux version 10.0 CD-ROM
image file is named \url{ttylinux-i486-10.0.iso.gz}; decompress it and mount it
via loop device with the following commands, substituting your actual CPU
\url{x86_64} or \url{i686}, if needed.

\begin{lstlisting}
	mkdir -p mnt/ttylinux
	gunzip ttylinux-i486-10.0.iso.gz
	mount -t iso9660 -o loop ttylinux-i486-10.0.iso mnt/ttylinux
\end{lstlisting}

The two files needed from the CD-ROM are the ttylinux root file system,
\url{boot/filesys.gz}, and the Linux kernel, \url{boot/vmlinuz}. You
can, of course, use a different Linux kernel, following the ttylinux custom
kernel requirements described in section \ref{customkernel}.

There are two ways to install ttylinux for booting, one is to have ttylinux
boot with the root file system in RAM disk, the other is to have ttylinux boot
with the root file system directly on the hard drive.

{\bf Install a ttylinux to Boot Using RAM Disk}

Copy the ttylinux file system \url{filesys.gz} image and the desired Linux
kernel into your boot files directory; probably, this directory is \url{/boot}.
After copying the two files, unmount the loop device with the following command.

\begin{lstlisting}
	umount -d mnt/ttylinux
\end{lstlisting}

These two files, the kernel and the file system image, can have names other
than the file names from the ttylinux CD-ROM. For this example the file names
are changed from the names on the CD-ROM: the compressed ttylinux file system
image file is called \url{ttylinux-filesys.gz}, the Linux kernel is called
\url{ttylinux-vmlinuz} and the boot dirctory is \url{/boot}. Add the following
section to \url{/etc/lilo.conf}:

\begin{lstlisting}
image = /boot/ttylinux-vmlinuz
	label  = ttylinux
	initrd = /boot/ttylinux-filesys.gz
	root   = /dev/ram0
	read-only
\end{lstlisting}

Run the \url{LILO} boot loader installer by typing \url{/sbin/lilo}. The next
boot will have the option of selecting \url{ttylinux} at the \url{LILO} boot
prompt.

{\bf Install a ttylinux to Boot with File System on a Hard Drive}

A hard drive partition, or a flash drive partition, of at least 8 MB is needed
to install ttylinux. For this example ttylinux is being installed on drive
partition device /dev/hda8 and the kernel and file system files are available
via the loop device instructions above. A loop device also is used to mount the
ttylinux file system image file.

\begin{lstlisting}
	cp mnt/ttylinux/boot/filesys.gz filesys.gz
	gunzip filesys.gz
	mkdir -p mnt/filesys
	mkdir -p mnt/newroot
	mount -t ext2 -o loop ./filesys mnt/filesys
	mount -t ext2 /dev/hda8 mnt/newroot
	cp -a mnt/filesys/* mnt/newroot
	cp mnt/ttylinux/boot/vmlinuz mnt/newroot/boot/ttylinux-vmlinuz
	umount -d mnt/ttylinux
	umount -d mnt/filesys
\end{lstlisting}

The new ttylinux root file system is still mounted; it needs to be customized
before booting. Configuration is described in the following section
\ref{configuration}; it includes a description of a \url{LILO} configuration
for booting the new ttylinux installation. After configuration unmount
mnt/newroot.

\subsubsection{Manual Installation -- System Configuration}
\label{configuration}

This section covers the minimum configuration needed to run ttylinux. More
system configuration can be done; see the system guide, section \ref{sysguide},
below for information.

The configuration files and options described in this section are present in a
ttylinux system installed from the bootable CD-ROM. Following the installation
instructions above, the file system is in \url{mnt/newroot}, that is the
example starting point for the following configuration descriptions.

{\bf /etc/fstab}

\url{/etc/fstab} needs to have the correct device for the root directory, the
manually installed ttylinux \url{/etc/fstab} still specifies a RAM disk device
for the root directory. Change the RAM disk device, /dev/ram0, to be the disk
partition device in which the ttylinux root file system was installed. In the
above example /dev/hda8 was used, so for that example the root directory in
\url{/etc/fstab} would be specified as:

\begin{lstlisting}
/dev/hda8     /     ext2     defaults     0 0
\end{lstlisting}

{\bf Boot Loader}

The boot loader needs to specify the ttylinux kernel and root file system
device. Following the installation instructions above, the \url{LILO}
configuration file \url{/etc/lilo.conf} would include the following. Note the
{\it initrd} specifier is removed and the {\it root} specifier is changed to
/dev/hda8.

\begin{lstlisting}
image = /boot/ttylinux-vmlinuz
	label = ttylinux
	root  = /dev/hda8
	read-only
\end{lstlisting}

{\bf Keyboard Map}

To use the current keyboard map from the Linux computer being used to install
ttylinux, use the following commands.

\begin{lstlisting}
	rm mnt/newroot/etc/i18n/kmap
	mnt/newroot/bin/dumpkmap >mnt/newroot/etc/i18n/kmap
\end{lstlisting}

{\bf Timezone}

The best way to set the ttylinux timezone is to use the boot parameters as
described in section \ref{bootparms}. This needs to be done only one time for
an installed ttylinux system as this boot option becomes persistent.

{\bf Dial-up Network Information}

ttylinux does not {\it directly} support dial-up networking with PPP and has no
support at all for ISDN. Previous versions of ttylinux did have PPP and ISDN
support; their package structure is being re-organized and they may return in
some later version of ttylinux.

ttylinux does have the PPP binaries: \url{/usr/sbin/pppd} and
\url{/usr/sbin/chat}, but currently it is up to you to configure and use them.

{\bf Unmount and Reboot}

Now unmount the newly installed partition.

\begin{lstlisting}
	umount mnt/newroot
\end{lstlisting}

And reboot to run the new ttylinux system.

\newpage
\section{System Guide}
\label{sysguide}

This section gives a short overview of the ttylinux system, its configuration
and some of the installed programs.

\subsection{Boot Parameters}
\label{bootparms}

Boot parameters are typed as a command line in the boot loader, before the
Linux kernel is loaded. You will see these options when you boot the ttylinux
CD-ROM. The \url{vga=<mode>} and \url{modules=<module>[,<module>]} boot
parameters are not used by {\bf i486} ttylinux.

\begin{lstlisting}
console=<ttyS*> ........... Use serial dev <ttyS*> for console login. The
                            kernel will use it for console output and ttylinux
                            will also put a getty login on it.
                            For <ttyS*> use one of ttyS0, ttyS1, ttyS2 or ttyS3.

vga=<mode> ................ Use VESA frame buffer with <mode>.  Try vga=ask
                            to get a list of modes support by the kernel.
                          vga=0x301 640x480   8-bit | vga=0x161 1152x864   8-bit
                          vga=0x311 640x480  16-bit | vga=0x163 1152x864  16-bit
                          vga=0x303 800x600   8-bit | vga=0x307 1280x1024  8-bit
                          vga=0x313 800x600  16-bit | vga=0x31A 1280x1024 16-bit
                          vga=0x305 1024x768  8-bit | vga=0x31C 1600x1200  8-bit
                          vga=0x317 1024x768 16-bit | vga=0x31E 1600x1200 16-bit

quiet ..................... Don't print a bunch of stuff while booting, but do
                            show anomalies and errors.

enet ...................... Startup networking for Ethernet interfaces found by
                            the kernel. DHCP is used. Any started interface will
                            be eth0, eth1, eth2 or eth3.

nofsck .................... Do not run fsck on any file systems.

nosshd .................... Do not start sshd or make ssh keys; recommended for
                            any CPU slower than 1 Ghz. A script that makes the
                            ssh keys will be left in the /root directory.

modules=<module>[,<module>] ... Load specific kernel module(s) named <module>.
                                Use this to load your sound card module, or
                                secure digital module (sdhci_pci if on
                                the PCI bus), etc.

hwclock=(local|utc) ....... The hardware (CMOS) clock keeps local or UCT time.

tz=<timezone> ............. Set timezone to <timezone> by setting the TZ
                            environment variable. Example: tz=GMT-8  See the
                            following URL for a complete description of TZ.
         http://www.gnu.org/s/libc/manual/html_node/TZ-Variable.html#TZ-Variable

host=<name.domain.tld> .... Set the hostname to <name.domain.tld>, which by
                            this example is a Fully Qualified Domain Name. You
                            can use a simple <name>.

login=<tty*,tty*,...> ..... Allow login on devices e.g., ttyS1, etc. Embedded
                            system might use this to put a getty login(s) on a
                            serial port(s).

nologin=<tty*,tty*,...> ... Disallow login on devices e.g., tty1, tty2, etc.
                            Embedded system might use this to prevent a getty
                            login(s) on a virtual console(s).
\end{lstlisting}

If you've installed ttylinux and are using lilo, or some other boot loader,
ttylinux will still use these boot options even if the boot loader does not
show them.

\subsection{Basic Features}

Upon boot-up, ttylinux provides 6 text consoles for login. There are two
initial accounts: \url{root}, the administrator account, with password
\url{password}; \url{user}, a user account, with password \url{password}

The \url{syslogd} and \url{klog} daemons are running and logging kernel and
system messages to the file \url{/var/log/messages}.

The available text editor is \url{vi}; invoke it by typing \url{vi}
\url{/path/to/filename}. This version of \url{vi} is a minimal version provided
by \url{busybox}. Documentation and help for using \url{vi} is available in
many places on the web.

For manipulation of users, groups and passwords, the tools \url{adduser},
\url{addgroup}, \url{deluser}, \url{delgroup} and \url{passwd} are present.

If you have not changed the keyboard settings as outlined in the configuration
section, section \ref{configuration} above, ttylinux will use its default
keyboard settings. The default keyboard mapping is for a US keyboard.

The \url{inetd} super-server and the \url{dropbear} SSH server are running by
default. An FTP or TFTP server will be forked by \url{inetd} when receiving
either a connect from an FTP or TFTP client, respectively.

ttylinux has {\bf no} \url{telnet} server or client; the \url{dropbear} SSH
client is used to remotely log in to other hosts.

ttylinux includes a basic packet filtering firewall which is enabled by
default. Section \ref{firewaller} below describes the configuring the ttylinux
firewall.

\subsection{Bootup, Shutdown and System Configuration}
\label{startup}

On system bootup, the init process runs the \url{/etc/rc.d/rc.sysinit} script
to setup the system, such as setting the clock, system font, keyboard map and
checking the file systems. \url{rc.sysinit} also runs the
\url{/etc/rc.d/rc.local} script {\it and then} runs all the programs in the
\url{/etc/rc.d/rc.startup} directory, all with the command line parameter
{\tt start}.

For ttylinux systems installed by the \url{/sbin/ttylinux-flash} script, the
system startup script \url{/etc/rc.d/rc.sysinit} attempts to find and mount the
drive that is booted from and then run a script on that drive. Remember: this
ttylinux system is a RAM disk system, changes made to files are not retained
for the next boot because the file system is in RAM. This feature of mounting
the booted drive and running a startup script provides the user with a
persistent boot-time startup configuration that is retained from one boot to
the next. This user-maintainable startup script on the boot drive is in the
\url{config/ttylinux} directory; a default version of this script is put onto
the drive when the \url{/sbin/ttylinux-flash} script is run. This script is a
very convenient place to configure your particular network interface for
ttylinux startup.

On system shutdown, the script \url{/etc/rc.d/rc.sysdone} runs. This script
runs all the programs in the \url{/etc/rc.d/rc.shutdown} directory {\it and
then} runs the \url{/etc/rc.d/rc.local} script, all with the command line
parameter {\tt stop}.

All the programs in \url{/etc/rc.d/rc.startup} and \url{/etc/rc.d/rc.shutdown}
are symbolic links that reference actual shell scripts or binary programs; they
are run in the ASCII order of their symbolic link file names. These symbolic
links are named with leading numbers to help control their ordering e.g.,
\url{10.network} is the symbolic link the the network startup program. The
actual programs are in \url{/etc/rc.d/init.d}. Removing a symbolic link
disables the program from starting up. These programs typically are shell
scripts; they are commonly called {\it initscripts}.

Initscripts can be interactively invoked. The following command runs the network
script \url{/etc/rc.d/init.d/syslog} with the command line option \url{stop}.

\begin{lstlisting}
	service syslog stop
\end{lstlisting}

All scripts use the command line options \url{start}, \url{stop}, \url{reload},
\url{restart} and \url{status}. They print a list of supported options if they
are called with no option present.

The initscripts define the basic ttylinux bootup system configuration. The
initscripts are configurable to an extent; thus the bootup configuration is
configurable to an extent. The bootup system configuration is specified in
ASCII text files in the \url{/etc/sysconfig} directory; this directory is
intended to have only files that are read by the various initscripts. All files
read by initscripts for configuration options should reside in
\url{/etc/sysconfig}.

There are two files in \url{/etc} that describe your ttylinux build-time
configurations. \url{/etc/ttylinux-host} is a text file that describes
something about the host architecture that built your ttylinux distribution.
\url{/etc/ttylinux-target} is a text file that describes some things about the
architecture of your running ttylinux distribution.

\subsection{Shell Environment}

The default shell used by ttylinux is GNU bash. The shell environment of
aliases and variables is in \url{/etc/profile}; view this file after login to
become familiar with the default shell environment.

Upon login, the \url{PATH} environmental variable has the following paths in
the order listed.

\begin{lstlisting}
	/bin
	/usr/bin
	/sbin
	/usr/sbin
\end{lstlisting}

Put additional, or overriding, shell environment in scripts in the
\url{/etc/profile.d} directory; do not change \url{/etc/profile} in order to
avoid losing changes when updating ttylinux.

\subsection{Using Dropbear for SSH}
\label{dropbear}

SSH, or secure shell, is a protocol for remote login with an advantage over
telnet being that it can use public key authentication instead of passwords.
Another advantage over the telnet protocol is that plain text is not
transfered; the data sent between the host connections is encrypted.

\url{dropbear} is a small SSH v2 server and client package. Keys are generated
and the server is started on system bootup by default, unless either the
ttylinux \url{dropbear} starup script detects the CPU is slower than 1 GHz or
the \url{nosshd} boot options was specified.

\url{dropbear} allows password and public key authentication. Public key
authentication can use DSS and RSA keys and works with keys generated by the
popular OpenSSH package. Having a public key from OpenSSH in the file
\url{.ssh/authorized_keys} should allow secure login from the machine that has
the corresponding private key. The permissions on the \url{.ssh} directory must
not include group or other write permission, otherwise \url{dropbear} will
refuse public key authentication.

The SSH client program is called \url{dbclient}. It is different from the
server in that it cannot use keys in OpenSSH format. You can use the
\url{dropbearconvert} program to convert an OpenSSH format key for use by
\url{dbclient} or you can use \url{dropbearkey} to create a new key.

To convert an OpenSSH key stored in \url{~/.ssh/id_rsa}, do:

\begin{lstlisting}
	dropbearconvert openssh dropbear \
		~/.ssh/id_rsa ~/.ssh/id_rsa.db
\end{lstlisting}

The new key will be stored in \url{~/.ssh/id_rsa.db}. You can use the \url{-i}
switch to \url{dbclient} to make it use your new key for authentication. The
public key part of the old OpenSSH key can be used as-is for pasting into your
\url{~/.ssh/authorized_keys} file. Conversion is only needed for the private
key.

To create a new RSA key to store in \url{~/.ssh/id_rsa.db}, you can use the
following command:

\begin{lstlisting}
	dropbearkey -t rsa -f ~/.ssh/id_rsa.db
\end{lstlisting}

The public key part of the new key will be printed to the screen. You can put
it into the \url{~/.ssh/authorized_keys} file on all machines where you want to
be able to login using your new private key stored in \url{~/.ssh/id_rsa.db}.
You can create a DSS key instead of an RSA key by using \url{-t} \url{dss}
instead of \url{-t} \url{rsa}. Should you lose the public key, you can always
get it back by using the private key and the \url{-y} switch to
\url{dropbearkey}:

\begin{lstlisting}
	dropbearkey -y -f ~/.ssh/id_rsa.db
\end{lstlisting}

If you want to use \url{scp} to copy files from another machine, the standard
\url{scp} program from OpenSSH is included with \url{dropbear} and ttylinux.

\subsection{Using an Ethernet Network}

ttylinux is ready to use Ethernet networking. DHCP will be used when starting
up the Ethernet network, unless configured otherwise.

The Ethernet network interface configuration is specified in the text file:
\begin{lstlisting}
	/etc/sysconfig/network-scripts/ifcfg-eth0
\end{lstlisting}

This file has specification in the form of "ITEM=value". Edit this file to set
the proper Ethernet interface IP addresses, change the Ethernet DHCP usage and
to enable Ethernet networking. To enable Ethernet networking, the line
\url{ENABLE=no} must be changed to \url{ENABLE=yes}. To disable DHCP, the line
\url{DHCP=yes} must be changed to \url{DHCP=no}.

After configuring the Ethernet network interface, restart the networking
subsystem with the following command.

\begin{lstlisting}
	service network restart
\end{lstlisting}

See the description of the \url{/sbin/sysconfig} script in section
\ref{sysconfig} for scripted help in setting up the Ethernet network interface
configuration.

The Ethernet network interface, commonly referred to as eth0, can be started
and stopped independently from the entire network subsystem with the following
commands.

Startup eth0 with:
\begin{lstlisting}
	ifup eth0
\end{lstlisting}

Shutdown eth0 with:
\begin{lstlisting}
	ifdown eth0
\end{lstlisting}

{\bf Help! I can use only IP addresses and not domain names.}\\
{\bf /etc/resolv.conf}

If you use only static IP addresses and no DHCP, as specified in your 
/etc/sysconfig/network-scripts/ifcfg-eth* files, then you probably have no
\url{/etc/resolv.conf} file and no domain name resolution. In this case only
IP addresses will work, such as "ping 212.123.44.77", and domain names will
{\bf NOT} work, such as "ping sun.com".

In this case, make your own \url{/etc/resolv.conf} file with these contents:
\begin{lstlisting}
# OpenDNS Servers
nameserver 208.67.222.222
nameserver 208.67.220.220
\end{lstlisting}

The following bash commands will do this for you:
\begin{lstlisting}
	rm -f /etc/resolv.conf
	>/etc/resolv.conf
	echo "# OpenDNS Servers"         >>/etc/resolv.conf
	echo "nameserver 208.67.222.222" >>/etc/resolv.conf
	echo "nameserver 208.67.220.220" >>/etc/resolv.conf
\end{lstlisting}

\subsection{Using the Firewall}
\label{firewaller}

The ttylinux firewall script sets the firewall to drop all new network input
except for the ports explicitly specified in the firewall configuration file
\url{/etc/firewall.conf}. The default firewall configuration specified in
\url{/etc/firewall.conf} allows connections for FTP, TFTP, SSH, HTTP and the
unprivileged UDP ports 1024 through 65535. The \url{/etc/firewall.conf}
firewall configuration file has a very simple syntax that includes comments;
the default file contains comments explaining its syntax and should be easy to
understand and update.

Outgoing traffic is not firewalled at all and there is no configuration file
for controlling outgoing traffic.

\subsection{Using NFS}

ttylinux can be an NFS client; NFS versions 2 and 3 are supported.

tylinux is prepared to automatically mount NFS entries you add to the
\url{/etc/fstab} file.

Example manual commands to mount a NFS directory:
\begin{lstlisting}
	mount -t nfs -o nolock,rw,vers=2 <nfs server>:<exported dir> /mnt/nfs
	mount -t nfs -o nolock,rw,vers=3 <nfs server>:<exported dir> /mnt/nfs
\end{lstlisting}

\subsection{Using Dialup}

ttylinux does not {\it directly} support dial-up networking with PPP and has no
support at all for ISDN. Previous versions of ttylinux did have PPP and ISDN
support; their package structure is being re-organized and they may return
in a later version of ttylinux.

ttylinux does have the PPP binaries: \url{/usr/sbin/pppd} and
\url{/usr/sbin/chat}, but currently it is up to you to configure and use them.

\subsection{Package Management}
\label{pacmanger}

Package management is handled by \url{pacman}; it is invoked from the shell
by typing its name, \url{pacman}. Use \url{pacman} to install and remove
packages, and to query the local database of installed packages and files. When
the network is up \url{pacman} can query, download and install packages from
appropriate repositories. \url{http://ttylinux.net/} currently is the only
known ttylinux package repository. \url{pacman} also can make ttylinux
packages, which may be handy as \url{pacman} is a bash script that can run on
any Linux system.

ttylinux packages are \url{tar} archives compressed with \url{bzip2}. All the
packages that come with the ttylinux distribution are available in the CD-ROM
ISO image; this is for reinstalling any packages that may have been removed
from a ttylinux system.

A list of hostnames to be used as ttylinux repositories may be kept in file
\url{/etc/ttylinux-repo}; however this file does not need to exist. A package
repository hostname can be given to any pacman command that accesses a ttylinux
package repository, and \url{ttylinux.net} is allways the default if no package
repository is given. If \url{/etc/ttylinux-repo} exists and has hostnames in
it, \url{ttylinux.net} will be search first if no package repository is given
on the pacman command line. You are not likely to use the
\url{/etc/ttylinux-repo} file, nor the repo option in general, as there
currently is no known ttylinux package repository.

\url{pacman} uses directory \url{/usr/share/ttylinux} as a database location.
In this directory, ttylinux has one file per installed package; each file
lists of all the file pathnames that belong that package. \url{pacman} makes
and removes these package database files as needed. Also in the
\url{/usr/share/ttylinux} directory will be similar repository cache files, one
each for each ttylinux repository that \url{pacman} uses.

{\bf Pacman Usage}\\
Some of the information from "pacman --help":
\begin{lstlisting}
Usage: pacman [option ...] operation name [name ...]

Options:
  --repo=<name>  refer to a particular external repository
  --vers=<num>   download for ttylinux version V<num>
  -v|--verbose   verbose operation

Operations:
  -h|--help          display this option summary
  -d|--download      download package files
  -e|--erase         remove packages
  -i|--install       install package files
  -m|--make          make a package
  -qa|--query-all    list all installed packages
  -qf|--query-file   show package that has file
  -ql|--query-list   list files from named package
  -qr|--query-repo   list packages in external repositiory
\end{lstlisting}

The \url{pacman} command line above shows a specific order for {\it options},
{\it operations} and {\it names}; however, they actually can be arraged in any
order.

{\bf Options}\\
Options can be repeated. If multiple conflicting options are given, the last
one is used and the others are silently ignored. Each option applies only to
some operations.

{\bf Operations}\\
One operation must be supplied, and only one operation is allowed; all other
uses of \url{pacman} display a help summary.

{\bf Package Names}\\
There are two kinds of package names used with \url{pacman}.

Package {\it download}, {\it installation}, {\it make} and
{\it query repo} operations refer to actual binary package file names.
These files are \url{tar} archives compressed with \url{bzip2}.

Package {\it erase}, {\it query all}, {\it query file} and {\it query list}
operations refer to package names without the CPU architecture and {\tt .tbz}
suffix. This shorter name conceptually refers to the package as its files are
installed in various directories; it is not a name of a literal package file.
With the {\it erase} and {\it query list} operations the name is given on the
\url{pacman} command line. With the {\it query all} and {\it query file}
operations the name is listed as output from \url{pacman}.

\subsubsection{Using pacman with ttylinux}

{\bf Package Download}

Examples
\begin{lstlisting}
	pacman --download bash-3.2.48-x86_64.tbz lilo-22.8.src-x86_86.tbz
	pacman -d bash-3.2.48-x86_64.tbz --repo=ttylinux.org
	pacman -d bash-3.2.48-x86_64.tbz lilo-22.8.src-x86_86.tbz --vers=9.1
	pacman --repo=palooka.net --download --vers=9.1 bash-3.2.48-x86_64.tbz
\end{lstlisting}

Use the \url{-d} or \url{--download} option to download a package from a
repository via the network. The package name given to the command is the actual
name of the package file. Multiple package names can be used to download
multiple packages.

If no repository is given with the \url{--repo=} option, then all known
repositories are searched. If the \url{--repo=} option is used, then only the
given repository is search. The first package found matching the given package
name is downloaded if there is a ttylinux version match, and then the download
command stops; there is no further package search after a download attempt.

The version of the running ttylinux from which the download command is given
must match the ttylinux version for a matching package name, otherwise the
package is skipped and the download command continues searching the repository.
Matching package names are displayed with their ttylinux version.

The \url{--vers=} option is used to override the running ttylinux version. For
example, with this option you can download a package generated for ttylinux-9.1
while running ttylinux-9.3, which in many cases is a valid option; furthermore,
this option must be used with pacman when running pacman from a non-ttylinux
host. See section \ref{foreignpacman} for using pacman on a non-ttylinux host.

{\bf Package Erase}

Examples
\begin{lstlisting}
	pacman --erase e2fsprogs-1.41.3
	pacman -e e2fsprogs-1.41.3 bash-3.2.48
\end{lstlisting}

Use the \url{-e} or \url{--erase} option to remove an installed package's files
and remove the package database file. The {\it name} is the name of the package
as shown by the pacman query operations; this is not the name of the actual
binary package file.  Multiple package names can be used to remove multiple
packages.

\url{pacman} will show the package name ask to continue to remove the given
package, and it will always list all the removed files.

Use the \url{-v} or \url{--verbose} option to get verbose output during package
removal.

{\bf Package Install}

Examples
\begin{lstlisting}
	pacman -i packages/bash-3.2.48-i486.tbz
	pacman -i bash-3.2.48-i486.tbz e2fsprogs-1.41.3-i486.tbz
	pacman --install --repo=ttylinux.net bash-3.2.48-i486.tbz
	pacman --repo=ttylinux.net --install --vers=9.1 bash-3.2.48-i486.tbz
\end{lstlisting}

Use the \url{-i} or \url{--install} option to install a package by giving the
package file name. The package name can be a pathname; the actual package file
must be as named in the pacman command unless the \url{--repo=} option is used.
Multiple package names can be used to install multiple packages.

To install from a repository use the \url{--repo=} option to give the hostname
of the ttylinux package repository. Matching package names are displayed with
their ttylinux version, but if the package's ttylinux version does not match
the running ttylinux then the package is not installed. For the package
installation command, the \url{--vers=} option is used only with the
\url{--repo=} option to override the running ttylinux version.  For example,
with this option you can install a package generated for ttylinux-9.1 while
running ttylinux-9.3, which in many cases is a valid option.

{\bf Package Make}

Examples
\begin{lstlisting}
	pacman -m dropbear-0.52-i486.tbz
	pacman -make bash-3.2.48-i486.tbz e2fsprogs-1.41.3-i486.tbz
\end{lstlisting}

Use the \url{-m} or \url{--make} option to make ttylinux packages. This is a
very difficult command to use. This command can be used on ttylinux but it is
intended to be used on a non-ttylinux host to assemble ttylinux packages.

This command looks for a package database file. The database file name is the
package name without the CPU architecture and {\tt .tbz} suffix. For the
dropbear example above, the database file name will be
{\tt pkg-dropbear-0.52-FILES} and that database file must be in the
{\tt /usr/share/ttylinux} directory.

Use the \url{-v} or \url{--verbose} option to get verbose output during package
making.

See section \ref{foreignpacman} for using pacman on a non-ttylinux host to make
ttylinux packages.

{\bf Query All} {\it List the Installed Packages}

Examples
\begin{lstlisting}
	pacman -qa
	pacman --query-all
\end{lstlisting}

Use the \url{-qa} or \url{--query-all} option to see the list of all installed
packages. This command shows the general package names, not the actual binary
package file names. The package names shown by this command are the package
names to use with the \url{pacman} erase command.

{\bf Query File} {\it Show the Package to which a File Belongs}

Examples
\begin{lstlisting}
	pacman -qf /bin/login
	pacman --query-file /bin/ls /usr/bin/pacman
\end{lstlisting}

Use the \url{-qf} or \url{--query-file} option to find out which package a
file belongs to. If the given file name does not actually exists on the system
there will be no output from \url{pacman}. If the given file name is not in
an installed package, then there will be no output from \url{pacman}.

{\bf Query List} {\it List the Files of an Installed Package}

Examples
\begin{lstlisting}
	pacman -ql e2fsprogs-1.41.3
	pacman --query-list e2fsprogs-1.41.3 bash-3.2.48
\end{lstlisting}

Use the \url{-ql} or \url{--query-list} option to list all files in the given
package.

{\bf Query Repository}

Examples
\begin{lstlisting}
	pacman -qr
	pacman --query-repo calc e2fsprogs-1.41.3-i486.tbz
	pacman -qr --repo=waldo.net
\end{lstlisting}

Use the \url{-qr} or \url{--query-repo} option to list packages in an external
ttylinux package repository.

When the \url{--repo=} option is used, then only that repository is used;
otherwise the \url{ttylinux.net} repository {\it and} all the repositories
named in the \url{/etc/ttylinux-repo} file are used.

The package name is the actual name of the binary package file; this is the
same package name to use when installing a package.

When no package name is given all packages in the appropriate repositories are
listed. This can be bothersome when looking for a particular package, as the
name of the package may scroll way off the screen as the repository is listed.
Use the name of a package, or a partial name, with this command to limit the
output to the package of interest. The package names given with this command
can be shortened; the partial name must be from the beginning of the package
name, with no gaps or wildcards. For example, a repository query for package
name \url{calc} will find and list only the binary package files beginning with
\url{calc}, listing the actual package names such as
\url{calc-2.12.4.0-i486.tbz} and \url{calc-2.12.4.0-x86_64.tbz}. All
appropriate repositories will be search in this manner.

\subsubsection{Using pacman on a non-ttylinux Host}
\label{foreignpacman}

{\bf Package Making Issues}

The intended use of \url{pacman} on non-ttylinux hosts is for assembling
ttylinux packages.

Package making seems an easy task as ttylinux packages are \url{tar} archives
compressed with \url{bzip2}. A complicating factor in using \url{pacman} to
make ttylinux packages is that the package making operation is an exact inverse
of package installation, and there is a controlling list of files that comprises
the files in the package. This controlling list of files is the database file
associated with the package.

A package database file is created by the \url{pacman} package installation
operation; it is read by the \url{pacman} package making operation.

All the package database files are in the \url{/etc/share/ttylinux} directory,
which your non-ttylinux host probably does not have, nor should you want that
directory on your non-ttylinux host.

Also, package installation puts files in directories all over your system, and
for the inverse operation of package making you do not want to get files from
directories all over your system. Putting files all around your system's
directories in order to make a ttylinux package is a bothersome at best, and
risky at worst.

Ideally, you would have a staging area, with the ttylinux files for which you
want to create a package, under a single convenient private directory. And
the package database file in a convenient private directory.

{\bf Alternate Directories}

\url{pacman} can use a user-specified root directory from which it gets the
files to make a package, and it can use a user-specified directory to look for
the package database file. These user-specified directories, file root and
package database directories, are used by {\bf all} the \url{pacman} operations
except package {\it download} and {\it query repo} operations. With this
capabilities, \url{pacman} can install, query and make packages all with an
alternate private root directory, and using an alternate private directory for
managing the package database files.

The appropriate way to use \url{pacman} on a non-ttylinux host is to set the
\url{pacman} file root and package database directories to your own alternate
private directories. This is done by using environment variables. It may be
convenient to use shell scripts in which the environment variables that
control the user-specified directories are set, and sequences of \url{pacman}
commands work within this environment.

There are three environment variables that customize \url{pacman} behaviour.

\begin{lstlisting}
PACMAN_FILES_ROOT_DIR  This sets the root directory that pacman uses to install
                       files and also to look for files when making packages or
                       removing packages. The default that pacman uses when
                       this is not set is the system's root directory, /.

PACMAN_PACKAGE_DB_DIR  This sets the directory in which to refer to, make or
                       remove the package database files. The default that
                       pacman uses when this is not set is /usr/share/ttylinux.

PACMAN_REPO_CACHE_DIR  This sets the directory that pacman uses to make
                       repository cache files for all repository operations.
                       The default that pacman uses when this is not set is
                       /usr/share/ttylinux.
\end{lstlisting}

An example usage of these environment variables for using \url{pacman} on a
non-ttylinux host is:
\begin{lstlisting}
#!/bin/bash

# Script to merge the bash and busybox packages into one single package.
# The pacman script must be in the $PATH.

export PACMAN_FILES_ROOT_DIR=$(pwd)/p_root
export PACMAN_PACKAGE_DB_DIR=$(pwd)/p_root/usr/share/ttylinux
export PACMAN_REPO_CACHE_DIR=$(pwd)/p_root/usr/share/ttylinux

# Remove everything created by this script so that this script is repeatable.
#
rm -rf p_root
rm -rf new-stuff-i486.tbz

# Make an alternate root directory.
#
mkdir -p p_root/usr/share/ttylinux

pacman --repo=ttylinux.net --install --vers=9.1 bash-3.2.48-i486.tbz
#
# Now these two files are installed:
#      $(pwd)/p_root/bash
#      $(pwd)/p_root/sh -> bash
# And there is a package database file:
#      $(pwd)/p_root/usr/share/ttylinux/pkg-bash-3.2.48-FILES
# And there is a repository cache file:
#      $(pwd)/p_root/usr/share/ttylinux/repo-ttylinux.net

pacman --repo=ttylinux.net --install --vers=9.1 busybox-1.15.3-i486.tbz
#
# Now there are a bunch of files in:
#      $(pwd)/p_root/bin/
#      $(pwd)/p_root/sbin/
#      $(pwd)/p_root/usr/bin/
#      $(pwd)/p_root/usr/sbin/
#      <etc>
# And there is a package database file:
#      $(pwd)/p_root/usr/share/ttylinux/pkg-busybox-1.15.3-FILES
# And there is a repository cache file:
#      $(pwd)/p_root/usr/share/ttylinux/repo-ttylinux.net

# Merge the two packages and make a new single package.
#
cat $(pwd)/p_root/usr/share/ttylinux/pkg-bash-3.2.48-FILES \
    $(pwd)/p_root/usr/share/ttylinux/pkg-busybox-1.15.3-FILES \
    >$(pwd)/p_root/usr/share/ttylinux/pkg-new-stuff-FILES
pacman --make new-stuff-i486.tbz

# Show the new package binary file.
#
ls -hl new-stuff-i486.tbz

exit 0
\end{lstlisting}

\subsection{Using the sysconfig Script}
\label{sysconfig}

The \url{/sbin/sysconfig} shell script can be used to set, and to show, the
fields in various ttylinux system configuration files; it can set or show any
value for any "ITEM=value" line in any configuration file in the
\url{/etc/sysconfig} and \url{/etc/sysconfig/network-scripts} directories.

The following commands sets "ENABLE=yes" and "DHCP=yes" in the
\url{/etc/sysconfig/network-scripts/ifcfg-eth0} file.

\begin{lstlisting}
	sysconfig -nc ifcfg-eth0.enable=yes
	sysconfig -nc ifcfg-eth0.dhcp=yes
\end{lstlisting}

The "-nc" option in the above examples tells the sysconfig script to work on
files in the \url{/etc/sysconfig/network-scripts} directory. The second option
is in the form file.item=value.

To change the IP address of the Ethernet network interface, with 192.168.1.100
as the example IP address, with a netmask of 255.255.255.0 and standard subnet
gateway and broadcast addresses, use the following sequence of sysconfig script
commands.

\begin{lstlisting}
	sysconfig -nc ifcfg-eth0.ipaddress=192.168.1.100
	sysconfig -nc ifcfg-eth0.network=192.168.1.0
	sysconfig -nc ifcfg-eth0.netmaks=255.255.255.0
	sysconfig -nc ifcfg-eth0.gateway=192.168.1.1
	sysconfig -nc ifcfg-eth0.broadcast=192.168.1.255
\end{lstlisting}

Use "-sc" for the first option to the \url{sysconfig} script in order to work
with system configuration files in the \url{/etc/sysconfig} directory.

Use the following command to get complete, up-to-date help description directly
from \url{/sbin/sysconfig}
\begin{lstlisting}
	sysconfig --help
\end{lstlisting}

\subsection{Depricated and Legacy Items}

\subsubsection{Dial-up Networking}

ttylinux does not {\it directly} support dial-up networking with PPP and has no
support at all for ISDN. Previous versions of ttylinux did have PPP and ISDN
support; their package structure is being re-organized and they may return
in a later version of ttylinux.

ttylinux does have the PPP binaries: \url{/usr/sbin/pppd} and
\url{/usr/sbin/chat}, but currently it is up to you to configure and use them.

\newpage
\section{Add-ons}

Add-ons packages, such as \url{thttpd} a tiny web server, {\bf MIGHT} be
available at the ttylinux web site. There also are links to any known off-site
ttylinux add-on resources. New add-ons submitted to ttylinux will be considered
for inclusion at the web site.

The ttylinux package manager, \url{pacman}, has the ability to install add-ons
directly from the ttylinux web site. Section \ref{pacmanger} describes the
\url{pacman} ttylinux package manager.

\newpage
\section{Contact and Help}

Reporting bugs in ttylinux and its documents is appreciated. For bug reports,
suggestions, or anything else about ttylinux that you think is important, feel
free to contact me. You can reach me by email at:

\hspace{0.5in}Douglas Jerome \url{<douglas@ttylinux.org>}

There is a web-based forum that is active from time to time; it is active when
this was written, April 2010, and is intended to be active as long as
minimalinux is supporting ttylinux, barring spammer abuse.

\hspace{0.5in}\url{http://www.minimalinux.org/forum/}

Help may be available on irc, although it is very low bandwith and usually more
appropriate for inane banter.

\hspace{0.5in}\url{irc.freenode.net} \#\url{ttylinux}

\appendix

\newpage
\section{ttylinux-specific Commands Overview}

Separate from the initscripts in \url{/etc/rc.d/initd} directory, the
following table lists the ttylinux-specific scripts intended to be available
for ttylinux root users.

\begin{center}
\begin{tabular}{l|l|l}
Script & Directory & Usage \\
\hline
ifdown             & /sbin    & Shutdown Ethernet Network Interface        \\
ifup               & /sbin    & Startup Ethernet Network Interface         \\
service            & /sbin    & Execute a script in \url{/etc/rc.d/init.d} \\
shutdown           & /sbin    & Reboot or Shutdown the System              \\
sysconfig          & /sbin    & Modify a System Configuration File         \\
ttylinux-flash     & /sbin    & Copy ttylinux to Flash Disk                \\
ttylinux-installer & /sbin    & Install ttylinux onto A Disk               \\
pacman             & /usr/bin & ttylinux Package Manager                   \\
\end{tabular}
\end{center}

\newpage
\section{Flash\_Disk\_Howto.txt}
\label{flashdiskhowto}

\begin{lstlisting}
How to Put ttylinux on a Flash Disk and Make it Bootable
Copyright (C) 2008-2010 Douglas Jerome <douglas@ttylinux.org>


FILE NAME

	$RCSfile: Flash_Disk_Howto.txt,v $
	$Revision: 1.11 $
	$Date: 2010/03/01 02:33:11 $

PROGRAM INFORMATION

	Developed by:	ttylinux project
	Developer:	Douglas Jerome, drj, <douglas@ttylinux.org>

FILE DESCRIPTION

	This document is a guide to putting ttylinux on a flash disk and making
	the it bootable.

CHANGE LOG

	28feb10	drj	Corrected for the latest CD-ROM layout and added timeout
			to the boot loaders to allow for boot options.

	19dec09	drj	Corrected the description of the two required flash
			drive directories. credit <legendre@nerp.net>

	01sep09	drj	Updated to be consistent with revised ttylinux-flash
			script and the CD-ROM directory and file structure.

	07dec08	drj	Changed some descriptions for using the syslinux
			executable program on the ttylinux CD-ROM.

	04dec08	drj	Added suggestions on mounting the CD-ROM and USB disk.

	22nov08	drj	Added failure path descriptions.  Finished testing the
			installation processes.  Added section numbers and the
			outline.

	22nov08	drj	Changed ram0 location from flash disk to /tmp.  Fixed
			the device referenced by the syslinux command.  Added
			description of lilo's anomalous behavior.  Fixed the
			fdisk usage in the description of boot problems.

	21nov08	drj	Finished and baselined first version for ttylinux.


                ------------------------------------------------


How to Put ttylinux on a Flash Disk and Make it Bootable


-- Document Outline --
1. Preface
2. Introduction
3. Lilo Method
4. Syslinux Method
5. Automated Help
6. Boot Problems


==========
1. Preface
==========

Caveat:  The syslinux method is known to work with syslinux-3.72.
Caveat:  Instead of booting ttylinux, your flash disk may become unusable, but
         that is not known to have happened.

Advice:  Read before doing; reading does not take very long.  Look at the end
         of this short document for problems and possible resolutions.


===============
2. Introduction
===============

Flash disks include USB disks which are often called flash drives, pen drives,
USB memory sticks, travel drives, etc.

This file describes two methods of copying ttylinux from its bootable CD-ROM
and putting it onto a flash disk that is also made bootable.  The syslinux
and lilo methods both can be done by ttylinux, but notice the syslinux program
that makes the flash disk become bootable is not in the ttylinux file system,
it is in the root directory on the ttylinux CD-ROM.  These methods probably
only make sense on a Linux system, particularly the lilo method.

You should save all your data on the flash disk to somewhere else and then
remove all files and directories from the flash disk.  Making a mistake in this
process can endanger any data on the flash disk.  Also, if the Linux kernel is
too far from the beginning of the flash disk memory it may not be bootable;
this has nothing to do with where the file name is in a directory listing or in
Windows explorer.

You can format the flash disk to be a Linux file system, but leaving a USB disk
in Windows format, probably vfat aka W95 FAT32, is very convenient.

Prerequisites:  Depending upon the method you use, you need to have privilege
                to write to the flash disk device e.g. /dev/sdc or to write to
                its mountable partition you want to use e.g. /dev/sdc1, and
                with the lilo method you need to create a device node.  It is
                therefore very likely you need to be root.

                You need to *know* the flash disk device node e.g. /dev/sdc and
                its mountable partition you want to use e.g. /dev/sdc1.  Read
                the previous sentence again, note the distinction between the
                disk and partition devices.

In the following descriptions <disk> and <partition> are used to represent
device nodes in the /dev directory.

<disk> is the device node of the entire flash disk e.g. sdc, in which case
       /dev/<disk> represents /dev/sdc.

<partition> is the device node of the mountable partition on the flash disk
            that you want to use to store the Linux kernel and ttylinux file
            system e.g. sdc1, in which case /dev/<partition> represents
            /dev/sdc1.

In the following descriptions, /mnt/flash references the mount point in your
file system to which the flash disk mounts.  Your actual mount point may be
different, substitute accordingly.

A USB disk partition probably should be mounted with the following mount
command.  The second command gives you the UUID of the mounted partition, it
may not work, but if it does then write down or otherwise save the UUID.

     $ mount -t vfat /dev/<partition> /mnt/flash
     $ blkid /dev/<partition>

/mnt/cdrom represents the location of the mounted CD-ROM in the following
descriptions.

Have the ttylinux boot CD-ROM in the CD-ROM drive and mount it.  The CD-ROM
should be mounted as type iso9660 e.g., mounted by the following command.

     $ mount -t iso9660 /dev/<disk> /mnt/cdrom

If you have an image of the ttylinux CD-ROM mounted via loopback device, or
have the files from the ttylinux CD-ROM in another directory, you can use that.

In the following descriptions there are example commands; they are prefixed by
a shell prompt of "$ ", and comments to shell commands begin with the shell
comment character '#".


==============
3. Lilo Method
==============

Warning:  After performing this method subsequent uses of the syslinux method
          may have no affect, or misboot with odd errors, or the lilo boot
          loader may remain on the flash disk and continue to boot the kernel.
          I've never seen the syslinux method work after using this lilo
          method.  There is a way to fix this; it is described at the end of
          the syslinux method.

Mount the flash disk.  The following description uses /mnt/flash to reference
the mount point of the flash disk.  Did you remember to first save everything
you want to keep off the flash disk and remove everything from it?  After
mounting the flash disk, create two new directories named "boot" and "config"
on the flash disk.

     $ mkdir /mnt/flash/boot
     $ mkdir /mnt/flash/config

The flash disk should now have nothing on it except the two empty directories
just made, /boot and /config.

Copy the ttylinux Linux kernel and ttylinux file system image from the CD-ROM
onto the flash disk; put them into the boot directory.

     $ cp /mnt/cdrom/boot/vmlinuz          /mnt/flash/boot/
     $ cp /mnt/cdrom/boot/filesys.gz       /mnt/flash/boot/
     $ cp /mnt/cdrom/config/ttylinux-setup /mnt/flash/config/ttylinux

You need a ram0 device node for lilo to reference during the boot installation.
If you don't have one in /dev then you need to make one somewhere; it is better
to NOT make one in /dev in the case your system uses udev.  You can make one in
/tmp with the following command.

     $ mknod -m 660 /tmp/ram0 b 1 0

A lilo configuration file is needed.  It is convenient to put it on the flash
disk in the boot directory; the file is /mnt/flash/boot/lilo.conf.  Use the
following example lilo.conf file, changing <disk> and </mnt/flash> and
</dev/ram0> to be the actual values.  Use ttylinux-flash=<UUID> ONLY if you got
the UUID when previously mounting the USB disk partition, replacing <UUID> with
the actual UUID value.

NOTE The location of the ram0 device is the actual one you want to use; if you
     didn't create one then it probably is /dev/ram0.

NOTE Everything between the dashed lines is the /mnt/flash/boot/lilo.conf file.

-------------------------------------------------------------------------------
boot = /dev/<disk>
disk = /dev/<disk> bios=0x80
map  = </mnt/flash>/boot/map

install     = menu
menu-scheme = Yb:Yk:kb:Yb
menu-title  = "LILO (LInux LOader) boot ttylinux"

compact
default = ttylinux
lba32
prompt
timeout = 150

image=</mnt/flash>/boot/vmlinuz
     append = "ro ttylinux-flash=<UUID>"
     label  = ttylinux
     root   = </dev/ram0>
     initrd = </mnt/flash>/boot/filesys.gz
     read-only
-------------------------------------------------------------------------------

After the lilo.conf file is correct, execute lilo to make the flash disk
bootable with these two commands.

     $ lilo -M /dev/<disk> mbr
     $ lilo -C /mnt/flash/boot/lilo.conf

There probably are many possible problems.  If there were no FATAL problems
reported from lilo, unmount and reboot the flash disk.

----------------
Possible Problem
----------------

Lilo may detect a partition problem and give you message like the following:

     Warning: boot record relocation beyond BPB is necessary: /dev/sdc
     Added ttylinux *
     Fatal: LILO internal error:  Would overwrite Partition Table

--------------------
Possible Resolutions
--------------------

If you have this problem you may want to do one of the following:

=> If you are using a USB disk then you can use a Windows-based USB boot disk
   tool; several are freely available.

=> Use a commercial partition tool to fix the flash disk partition table.

=> Use a different flash disk.


==================
4. Syslinux Method
==================

You need to have the syslinux executable program.  The root directory of the
ttylinux CD-ROM should have the syslinux executable program from syslinux-3.72.

Other syslinux sources: You may have it in your current linux distribution.  Or
you can get the latest version from
http://www.kernel.org/pub/linux/utils/boot/syslinux/ and after untarring it,
find the syslinux executable in the linux directory.

Caveat:  The syslinux method is only known by the author to work with
         syslinux-3.72; it probably works with newer versions and a few of the
         older versions.

Mount the flash disk.  The following description uses /mnt/flash to reference
the mount point of the flash disk.  Did you remember to first save everything
you want to keep off the flash disk and remove everything from it?  The flash
disk should now have nothing on it.

NOTE  The following lilo fixup also fixes many USB disks that do not properly
      boot.

NOTE  If you are doing this with a flash disk that previously was booting from
      a lilo boot loader e.g., you previously used the above lilo method, then
      perform this lilo operation before continuing:

      $ lilo -M /dev/<disk> mbr

Mount the flash disk.  The following description uses /mnt/flash to reference
the mount point of the flash disk.  Did you remember to first save everything
you want to keep off the flash disk and remove everything from it?  After
mounting the flash disk, create some new directories on the flash disk:

     $ mkdir /mnt/flash/boot
     $ mkdir /mnt/flash/boot/syslinux
     $ mkdir /mnt/flash/config

Copy the syslinux help message files from the CD-ROM onto the flash disk.  Copy
the ttylinux Linux kernel and ttylinux file system image files from the CD-ROM
onto the flash disk:

     $ cp /mnt/cdrom/boot/vmlinuz          /mnt/flash/boot/
     $ cp /mnt/cdrom/boot/filesys.gz       /mnt/flash/boot/
     $ cp /mnt/cdrom/boot/boot.msg         /mnt/flash/boot/syslinux/
     $ cp /mnt/cdrom/boot/help.msg         /mnt/flash/boot/syslinux/
     $ cp /mnt/cdrom/config/syslinux       /mnt/flash/config/syslinux
     $ cp /mnt/cdrom/config/ttylinux-setup /mnt/flash/config/ttylinux

A syslinux configuration file is needed.  It must be put on the flash disk in
the boot/syslinux directory; the file is /mnt/flash/boot/syslinux/syslinux.cfg.
Use the following example syslinux.cfg file.  Use ttylinux-flash=<UUID> ONLY if
you got the UUID when previously mounting the USB disk partition, replacing
<UUID> with the actual UUID value.  Everything between the dashed lines is the
/mnt/flash/boot/syslinux/syslinux.cfg file.
long.

-------------------------------------------------------------------------------
default ttylinux
display boot.msg
prompt  1
timeout 150

F1 boot.msg
F2 help.msg

label ttylinux
        kernel /boot/vmlinuz
        append initrd=/boot/filesys.gz root=/dev/ram0 ro ttylinux-flash=<UUID>
-------------------------------------------------------------------------------

Now make the flash disk bootable with syslinux; notice the partition device is
used, not the disk device.

     $ syslinux -d boot/syslinux /dev/<partition>

There probably are many possible problems.  If there were no problems, unmount
and reboot the flash disk.

----------------
Possible Problem
----------------

When executing the syslinux command you see an error message something like
"Cluster sizes larger than 16K not supported".

--------------------
Possible Resolutions
--------------------

Install a more recent version of syslinux.


=================
5. Automated Help
=================

It really is best to use the script described herein.

For the automated help described below, both the CD-ROM and the flash disk must
be mounted before executing the ttylinux-flash script.

There is a shell script in the ttylinux file system that does a variation of
the lilo and syslinux methods.  Backup anything you want to keep from your
flash disk before using the script.  The script is invoked with a command line
option telling it which method to use; guess which option does which.

ttylinux-flash --lilo     <CD-ROM path> <flash disk path> <flash disk device>
ttylinux-flash --syslinux <CD-ROM path> <flash disk path> <flash disk device>

The following command examples use the same conventions as above for the paths
and device nodes.

ttylinux-flash --lilo     /mnt/cdrom /mnt/flash /dev/<disk>
ttylinux-flash --syslinux /mnt/cdrom /mnt/flash /dev/<disk>

If you want to run this script from a Linux system other than ttylinux, then
run it from the ttylinux mounted at /mnt/cdrom, it will be
/mnt/cdrom/sbin/ttylinux-flash.


================
6. Boot Problems
================

General
-------

Some flash disks seem to have a boot problem, something wrong with their zero
block Master Boot Record (MBR).  Run fdisk on the disk device /dev/<disk> to
see if the Boot flag is set on the partition that has the Linux kernel,
/dev/<partition>.

     # Check for the Boot flag
     #
     fdisk -l /dev/<disk>

If the Boot flag is not set, use fdisk to toggle the bootable flag; the fdisk
command is 'a'.  The fdisk usage will look something like the following, if the
partition with the Linux kernel is 1.

     $ fdisk /dev/<disk>
     Command (m for help): a
     Partition number (1-8): 1
     Command (m for help): w

It also is best to use this lilo command, after having used fdisk to set the
partition bootable flag:

     $ lilo -M /dev/<disk> mbr

Strange Lilo Boot Errors
------------------------

If you get part of the word LILO and then nothing or a repeating sequence of
numbers or words, or if you get "Can't load operating system" or even nothing
at all: put the flash disk back into the computer from wich you where loading
it with ttylinux and try this lilo command:

     $ lilo -M /dev/<disk> mbr

Try bootable again after executing the above command; if the flash disk still
doesn't correctly boot, you may need to repeat either the lilo or syslinux
method of installing ttylinux.


[eof]
\end{lstlisting}

\end{document}

